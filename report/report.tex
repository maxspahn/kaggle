\documentclass[12pt,oneside,a4paper]{article}

\usepackage{./custom}

\begin{document}
% title (fold)
\begin{titlepage}
\begin{center}

% Upper part of the page. The '~' is needed because \\
% only works if a paragraph has started.

%\includegraphics[width=0.3\textwidth]{./img/logo}~\\[1cm]

\textsc{\LARGE Ecole Centrale Paris}\\[1.5cm]

\textsc{\Large Machine Learning, Chloe-Agathe Azencott}\\[0.5cm]

% Title
\HRule \\[0.4cm]
{ \huge \bfseries Report \\ Kaggle-Project \\[0.4cm] }

\HRule \\[1.5cm]

% Author
\begin{center} \large
\emph{Group: Leadersaucalme }\\
Max \textsc{Spahn}
\end{center}

\vfill

% Bottom of the page
{\large \today}

\end{center}
\end{titlepage}

% title (end)

\section{Preparation of the data}
The accesible data for the "kaggle"-competition is raw. The different parameters
are not normilized and has to be modified in order to apply the \textsc{sklearn}
algorithmes on it. The first modification we did was a circular permutation in
order to "normalize" parameters of time, such as \textsc{month, daytime, etc}.
The problem about time values is, that computers do not reckognize that January
and December are "close" to one each other. We arranged those parameters around
a clock, with an x-coordinate and a y-coordinate, so that with these two new
variables january and december get "close".

\end{document}
